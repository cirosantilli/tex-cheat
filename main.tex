\def\bs{$\backslash$}
% Brace Open
\def\bo{$\{$}
% Brace Close
\def\bc{$\}$}
% Computer code, until I find a better method.
\def\code#1{#1}
\def\section#1{{\bf #1}}
\def\subsection#1{{\bf #1}}
\def\subsubsection#1{{\bf #1}}

\section{Programming constructs}

  \TeX is Turing complete, and can do file IO or run shell commands, although those capabilities are disabled / restricted by default on most distros.

  This section contains a few constructs that remind me strongly of programming languages.

  \subsection{Assignment}

  \subsection{Registers}

  \subsection{Variables}

    Equal sign = and spaces when assigning are optional, unless their absence would generate ambiguity.

    To print their values use \bs number.

    \count255=1

    \number\count255

    \count255 2

    \number\count255

    %\count2553

  \subsection{new commands}

    The \code{new} family of commands reserves and names a resource to be used later, e.g. \code{newcount} or \code{newwrite}.

    They are the preferred way of using those resources to avoid clashes.

    There are 6 types of registers:

    \item{} newcount:  a number
    \item{} newdimen:  a dimension
    \item{} newskip:   glue
    \item{} newmuskip: muglue
    \item{} newtoks:   a token list
    \item{} newbox:    box

    There are also other types of resources:

    \item{} newread and newwrite: IO streams
    \item{} newfam:               font family
    \item{} newinsert:
    \item{} newlanguage:

    Usage:

    \newcount\x
    \x=1
    \number\x

  \subsection{Conditional}

  \subsection{if}

  \subsection{else}

  \subsection{fi}

  \subsection{ifx}

    \ifx\ifvar\undefined
      ifvar undefined
    \else
      \if\ifvar0
        ifvar = 0
      \else
        ifvar != 0
      \fi
    \fi

    \def\ifvar{0}

    \ifx\ifvar\undefined
      ifvar undefined
    \else
      \if\ifvar0
        ifvar = 0
      \else
        ifvar != 0
      \fi
    \fi

    \def\ifvar{1}

    \ifx\ifvar\undefined
      ifvar undefined
    \else
      \if\ifvar0
        ifvar = 0
      \else
        ifvar != 0
      \fi
    \fi

  \subsection{ifnum}

  \subsection{ifdim}

    Compare two numbers, only $=,$ $<$ and $>$ are possible operators.

    \newcount\a
    \newcount\b
    \a=1
    \b=2

    \ifnum\a<\b
      $1 < 2$
    \else
      $1 >= 2$
    \fi

    \ifnum\a<2
      $1 < 2$
    \else
      $1 >= 2$
    \fi

  \subsection{ifodd}

    Check if odd. There is no \code{ifeven} primitive.

    \ifodd1
      1 is odd
    \else
      1 is even
    \fi

  \subsection{relax}

    Do nothing. NOOP. When you need a command for some syntactical reason.

    \relax

  \subsection{undefined}

    Value of undefined macros.

    Normally, using an undefined macro leads to an error, but in some cases it may appear, e.g. when doing ifx.

  \subsection{number}

    Converts an internal number to a number on the output.

    \count255 = 1

    count255 = \number\count255

  \subsection{the}

    TODO. Like number but can format anything, e.g. dimen, etc.

    \count255 = 1

    count255 = \the\count255

    \dimen255 = 1cm

    dimen255 = \the\dimen255

  \subsection{Arithmetic}

    \subsubsection{Sum}

    \subsubsection{Subtraction}

    \subsubsection{advance}

      \newcount\x
      \x=1
      \advance\x by 1
      $1 + 1 = \number\x$

      \newcount\x
      \x=1
      \advance\x by -1
      $1 - 1 = \number\x$

    \subsubsection{multiply}

      \newcount\x
      \x=2
      \multiply\x by 2
      $2 * 2 = \number\x$

    \subsubsection{divide}

      \newcount\x
      \x=4
      \divide\x by 2
      $4 / 2 = \number\x$

  \subsection{def}

  \subsection{Macros}

    There are two types of backslash escapes: control words and control symbols.

    Control words are the regular macros like the ones that can be defined with \bs def. Eat up whitespace after them, so you need to preserve it with \bs or braces if wanted:

    \TeX without backslash

    \TeX\ with backslash

    \TeX{} with braces

    Control symbols backslashes followed by other non alphanumeric characters. They are often escapes for characters that have special meanings like percent \% for comments. They don't eat up the whitespace after them: \% after.

    \subsection{Macros vs primitives}

      Primitive commands can have any weird call syntax: sometimes the arguments are separated by English words like by in \code{divide by } or \code{read to}

      Macros have a fixed call syntax.

    \subsection{Macro call}

      You can call macro parameters wither with or without braces.

      If you don't use braces, it breaks at spaces.

      \def\f#1#2{#1 and #2}

      TODO what is wrong here:

      \f apple orange

      For arguments with spaces you need braces:

      \f{apple orange}{pear}

    \subsection{Infinite expansion loop}

      The following code would generate an infinite expansion loop: %\def\a{\a}\a

      Good possibility for a denial of service attack.

      The following leads to a "TeX capacity exceeded" error: %\def\a{\a{}b}\a

  \subsection{let}

    TODO vs def.

  \subsection{Groups}

    Group: {\bf bf}

  \subsection{input}

  \subsection{endinput}

    Paste content of given file at current location, and then process it as if it were there from the start.

    If an \code{endinput} is found, stop.

    \input input.texinput

  \subsection{File IO}

  \subsection{openout}

  \subsection{newwrite}

  \subsection{write}

  \subsection{newread}

  \subsection{read}

  \subsection{closeout}

    Write to a stream. \TeX\ as built-in file IO, besides \code{write18} which if enabled allows you to run arbitrary shell commands.

    Streams 0-15 are free to use, but it is best to allocate to them with \code{newwrite} and \code{newread}.

    Example:

    \newwrite\file
    \immediate\openout\file=io.tmp
    \immediate\write\file{From TeX!}
    \immediate\write\file{From TeX 2!}
    \immediate\closeout\file

    It is not possible to append to files: http://tex.stackexchange.com/questions/11796/how-can-i-open-a-file-in-append-mode

    \code{read} reads one line from a file, and stores it's value in a command: TODO get working

    \newread\file
    \immediate\openin\file=io.tmp
    \immediate\read\file to \fileline
    \fileline\par
    \immediate\read\file to \fileline
    \fileline\par
    \immediate\closein\file

    Some streams are special:

    18: send command to shell. This allows you to run an arbitrary command. Disabled by default on all distros: you need to give the \code{--shell-escape} option on invocation to activate it.

    E.g.:

    \immediate\write18{echo write18}

  \subsection{ifeof}

    True if at end of file.

    TODO get working

    \def\eofhuman#1{
      \ifeof#1
        At end of file.\par
      \else
        Not at end of file.\par
      \fi
    }

    \newread\file
    \immediate\openin\file=io.tmp
    \immediate\read\file to \fileline
    \fileline\par
    \eofhuman\file
    \immediate\read\file to \fileline
    \fileline\par
    \eofhuman\file
    \immediate\closein\file

  \subsection{immediate}

    Do an operation before further processing happens. E.g., there are things for which one has to wait to know, e.g. page numbers.

  \subsection{stdout}

  \subsection{message}

    Write to stdout.

    \message{Hello world from TeX!}

  \subsection{show}

    Print to stdout what the given command maps to.

    Debugging tool.

    TODO why is this giving compilation errors?

    %\def\a{asdf}
    %\show\a
    %\show\def

    Produces on terminal something like:

    %> \a=macro:
    %->asdf.
    %l.605   \show\a

    %> \def=\def.
    %l.606   \show\def

  \subsection{Security}

    Since \TeX\ is a programming language with IO possibilities, it is possible to exploit that to attack computers.

    The following attacks may be possible:

    - writing to local files to create a virus. This virus could also read local files, and then retrieve the information when the attacked user publishes something

    - read server files to prepare for a further attack

    - infinite loops to consume server CPU

    - write to an infinite amount of files to consume user / server disk

\section{Symbols}

  \aa

  \dag

  \TeX\ space

  \'e

  \"o

  i and j without dots: \i \j. Use when adding accents: \=\i

  char65 char66 = {\char65}{\char66}

  \section{Periods}

  Periods that don't follow upper case letters add some extra space after them assuming the end of a sentence.

  a. a

  A. a

  A\null. a

\section{Special characters}

  Escape the 10 special chars:

  \~{}: NBSP

  \#: macro arguments.

  \$: inline math

  \%: comments

  \^{}: TODO

  \&: alignment

  \_: TODO

  $\{$ $\}$: groups

  $\backslash$: macros

  There are also some special characters that appear as different things than their ASCII counterpart!

  \item{} pipe: | . Appears as an em-dash
  \item{} less and greater than < > Appear as an inverted ! and ?

  This weird stuff can be corrected with an encoding package in LaTeX.

  These are all the special chars that need to be escaped, the following don't: ` ! @ * () + = [  ] ? / , .

  \subsection{at sign @}

    While not special by default, the @ sign is used by convention LaTeX defined kernel commands that should not be overridden by end users. This works because @ is from category 11 which may appear in macro names.

    \def\@{at}

    \@

    http://tex.stackexchange.com/questions/6240/why-do-latex-internal-commands-have-an-in-them

\section{Styles}

  \subsection{Boldface}

  \subsection{bf}

    {\bf bold}

  \subsection{Italics}

  \subsection{it}

    {\it italic}

  \subsection{Teletype}

  \subsection{tt}

    {\tt teletype}

\section{rules}

  \hrule

  TODO vrule is giving me underfull vbox. How to avoid?

  \vrule height 1mm

\section{Positioning}

\section{Spacing}

  \bs break: start new page.

  \break

  \subsection{Control space}

    Spaces can be forced with a \bs space sequence.

    Multiple consecutive spaces can be added like that.

    a\ \ \ \ \ a

    aaaaaaa

\section{pageno}

  Current page number: \number\pageno

\section{Hyphens}

  Used to break words on long lines.

  \section{defaulthyphenchar}

    The default break char for a font.

    \bs number\bs defaulthyphenchar = \number\defaulthyphenchar = "{\char\defaulthyphenchar}"

  \section{hyphenchar}

    TODO

  \section{discretionary}

    Propose a possible word break on a large word and say what character \TeX\ can insert there if he wants to break.

    TODO: why the following line is not breaking at discretionary?

    %aaaaaaaaaaaaaaaaaaaaaaaaaaaaaaaaaaaaaaaaaaaaaaaaaaaaaaaaaaaaaaaaaaaaaaaaaa\discretionary{+}{=}{c}bbbbbbbbbbbbbbbbbbbbbbbbbbbbbbbbbbbbbbbbbbbbbbbbbbbbbbbbbbbbbbbbbbbbbb

  \section{discretionary hyphen}

    %\bs -. Same as \discretionary{\char\hyphenchar}{}{}

    %aaaaaaaaaaaaaaaaaaaaaaaaaaaaaaaaaaaaaaaaaaaaaaaaaaaaaaaaaaaaaaaaaaaaaaaaaa\-bbbbbbbbbbbbbbbbbbbbbbbbbbbbbbbbbbbbbbbbbbbbbbbbbbbbbbbbbbbbbbbbbbbbbb

\section{time}

  year:  \number\year

  month: \number\month

  day:   \number\day

  time:  \number\time. Minutes after midnight.

\section{Category code}

  Each character read from input has a category. There are 16 categories:

  0 : Escape character \bs

  1 : Beginning of group \bo

  2 : End of group \bc

  3 : Math shift \$

  4 : Alignment tab \&

  5 : End of line = ASCII return

  6 : Macro parameter \#

  7 : Superscript \^{}

  8 : Subscript \_

  9 : Ignored character = ASCII null

  10: Space and horizontal tab

  11: Letter a-zA-Z

  12: Other character (everything not listed above or below)

  13: Active character \~{} ASCII form feed

  14: Comment character \%

  15: Invalid character = ASCII delete

  It is possible to change the category of a character with \bs catcode.

\section{Document parts}

  This section documents commands which deal with high level document constructs like paragraphs and lists.

  \subsection{Paragraphs}

  \subsection{par}

    Paragraphs are created at double newline.

    It is also possible to create them with the \bs par command.

    Paragraph 1. \par
    Paragraph 2. \par
    Paragraph 3. \par

  \subsection{list}

  \subsection{List items}

    TeX has basic support for list items, but it is very basic and not intuitive at times.

  \subsection{Table}

    TODO

  \subsection{Images}

    TODO http://tex.stackexchange.com/questions/63789/inserting-an-image-in-plain-tex

  \subsection{verbatim}

    The main tricks are:

    \item{} change the catcode of special characters so that they are not special anymore
    \item{} preserve whitespace

    \subsubsection{string}

      TODO how does it work exactly? Seems to remove special meaning of special chars somehow.

      {\string \TeX. After.}

  \subsection{Syntax highlighting}

    Hell on Earth probably, even with LaTeX packages it is difficult.

  \subsection{Hyperlinks}

    TODO how?

    http://tex.stackexchange.com/questions/83847/the-errors-of-tex-with-hyperlinks

  \subsection{footline}

  \subsection{Footer}

  \subsection{headline}

  \subsection{Header}

    Content that will appear below and above every page.

    \headline={Header}

    \footline={Footer}

    Only apply starting to the first the next page after they are set.

  \subsection{Math}

    Plain \TeX\ was built with hardcore math typesetting in mind, and has become the de-facto standard mathematical plain-text notation.

    Inline math: $x^2$

    Displayed math:

    $$x^2$$

    Backslash math like \bs [ \bs ] is LaTeX only.

    \subsubsection{subscript}

      $$ ab_cd $$

      $$ ab_{cd} $$

    \subsubsection{superscript}

      $$ ab^cd $$

      $$ ab^{cd} $$

    \subsubsection{underline}

    \subsubsection{overline}

      $\underline x+1$

      $\underline{x+1}$

      $\overline x+1$

      $\overline{x+1}$

    \subsubsection{eqno}

    \subsubsection{leqno}

    \subsubsection{Equation numbers}

      Put an equation number on given display math.

      Only typesets the number: you have to increment them yourself.

      Only works for display math.

      \code{leqno} puts the number on the left.

      $$ x^2 \eqno{1} $$

      $$ x^2 \leqno{2} $$

    \subsubsection{over}

      Fractions.

      $$ab + cd \over ef + gh$$

      $$ab + {cd \over ef} + gh$$

    \subsubsection{atop}

      Like over but without separator.

      $$ab + cd \atop ef + gh$$

      $$ab + {cd \atop ef} + gh$$

    \subsubsection{above}

      Like over, but you can control the width of the separator:

      $$a \above 0pt b$$

      $$a \above 0.5pt b$$

      $$a \above 1pt b$$

      $$a \above 2pt b$$

      Therefore, over and atop can be defined in terms of this (0.4pt and 0pt respectively).

    \subsubsection{left}

    \subsubsection{right}

      Big delimiter characters.

      Every left must have a right, but delimiters don't need to match:

      $$\left( {a \over b} \right)$$

      $$\left( {a \over b} \right]$$

      $$\left( {a \over b} \right\}$$

      $$\left( {a \over b} \right>$$

      Omit one of the delimiters with a dot:

      $$\left( {a \over b} \right.$$


\section{every commands}

  There are a few command prefixed by every that add something before every construct of a type:

  \item{} everypar
  \item{} everydisplay
  \item{} everymath
  \item{} everycr
  \item{} everyjob

\section{Error messages}

  \subsection{Underfull hbox, vbox}

    Very hard to get rid of!

  \subsection{TeX capacity exceeded, sorry [input stack size=5000].}

    Stack overflow? Achieved with: %\def\a{\a{}b}\a
\end
